
\input{header.tex}

\usepackage{amsthm} % theorems
\usepackage{lettrine}
\usepackage{lastpage}
\usepackage{url}
\usepackage{mdframed}
\usepackage[T1]{fontenc}
%\usepackage[utf8]{inputenc}
\usepackage{times}
\usepackage{mathptmx}

\DeclareUnicodeCharacter{00A0}{~}  %get rid of "unicode char u8 not set up for use with latex"


%Options: Sonny, Lenny, Glenn, Conny, Rejne, Bjarne, Bjornstrup
\usepackage[Bjornstrup]{fncychap}


%%% -----------------------
%\usepackage[svgnames]{xcolor} % Required to specify font color

\newcommand*{\plogo}{\fbox{$\mathcal{PL}$}} % Generic publisher logo

%----------------------------------------------------------------------------------------
%	TITLE PAGE
%----------------------------------------------------------------------------------------

\newcommand*{\titleAT}{\begingroup % Create the command for including the title page in the document
\newlength{\drop} % Command for generating a specific amount of whitespace
\drop=0.1\textheight % Define the command as 10% of the total text height

\rule{\textwidth}{1pt}\par % Thick horizontal line
\vspace{2pt}\vspace{-\baselineskip} % Whitespace between lines
\rule{\textwidth}{0.4pt}\par % Thin horizontal line

\vspace{\drop} % Whitespace between the top lines and title
\centering % Center all text
\textcolor{red}{ % Red font color
{\Huge A Journey in Signal Processing}\\[0.5\baselineskip] % Title line 1
%{\Large de}\\[0.75\baselineskip] % Title line 2
{\Huge    with IPython}
} % Title line 3

\vspace{0.25\drop} % Whitespace between the title and short horizontal line
\rule{0.3\textwidth}{0.4pt}\par % Short horizontal line under the title
\vspace{\drop} % Whitespace between the thin horizontal line and the author name

{\Large \textsc{Jean-François Bercher}}\par % Author name


\vfill % Whitespace between the author name and publisher text
%{\large \textcolor{red}{\plogo}}\\[0.5\baselineskip] % Publisher logo
{\large \textsc{ESIEE-Paris}}\par % Publisher

\vspace*{\drop} % Whitespace under the publisher text

\rule{\textwidth}{0.4pt}\par % Thin horizontal line
\vspace{2pt}\vspace{-\baselineskip} % Whitespace between lines
\rule{\textwidth}{1pt}\par % Thick horizontal line

\endgroup}
%% ------------------------END TITLE PAGE

\makeatletter
\renewcommand{\@oddfoot}%
	{\hfil \upshape Page {\thepage}/\pageref{LastPage}}
\renewcommand{\@evenfoot}{\upshape Page {\thepage}/\pageref{LastPage} \hfil} %{\@evenfoot}
\makeatother

%%% Further colors and hyperref configuration
% Colors
\definecolor{webgreen}{rgb}{0,.5,0}
\definecolor{webbrown}{rgb}{.6,0,0}
\definecolor{webyellow}{rgb}{0.98,0.92,0.73}
\definecolor{webgray}{rgb}{.753,.753,.753}
\definecolor{webblue}{rgb}{0,0,.8}


\usepackage{hyperref}
\hypersetup{bookmarks,bookmarksnumbered,%bookmarksopen,
colorlinks,linkcolor=webbrown,filecolor=webgreen,citecolor=webgreen,
breaklinks=true,
hyperindex=true
urlcolor=webbrown,pagebackref,pdfpagemode=None,pdfstartview=Fit}
%%%----------------------------------------




%\title{\textbf{A Stroll with Random Signals and IPython}}
%\author{J.-F. Bercher}

%\title{\textbf{Régression linéaire}}
\begin{document}
%\maketitle

\thispagestyle{empty} % Removes page numbers

\titleAT % This command includes the title page


\def\E#1{\mathbb{E}\left[#1\right]}
\def\ta#1{\left<#1\right>}
\def\egalparerg{{\mathop{=}\limits_\mathrm{erg}}}
\def\egalpardef{\mathop{=}\limits^\triangle}
\def\E#1{\mathbb{E}\left[#1\right]}
\def\flecheTF{\rightleftharpoons }
\def\expo#1{\exp\left(#1\right)}
\def\dr#1{\mathrm{d}#1}
\def\d#1{\mathrm{d}#1}
\def\wb{\mathbf{w}} \def\sb{\mathbf{s}} \def\xb{\mathbf{x}}
\def\Rb{\mathbf{R}} \def\rb{\mathbf{r}} 
\def\TFI#1#2#3{{\displaystyle{\int_{-\infty}^{+\infty} #1 ~e^{j2\pi #2 #3} 
~\dr{#2}}}}
\def\TF#1#2#3{{\displaystyle{\int_{-\infty}^{+\infty} #1 ~e^{-j2\pi #3 #2} 
~\dr{#2}}}}
\def\mystar{{*}}


\def\R#1{\mathcal{R}\left\{#1\right\}}
\def\I#1{\mathcal{I}\left\{#1\right\}}
\def\tf#1{{\mathrm{FT}\left\{ #1 \right\}}}
\def\sha{w}   %{ш}
\def\sinc#1{{\mathrm{sinc}\left( #1 \right)}}
\def\rect{\mathrm{rect}}
\definecolor{lightred}{rgb}{1,0.1,0}
\def\myeqbox#1#2{{\fcolorbox{#1}{light#1}{$\textcolor{#1}{ #2}$}}}
\def\eqbox#1#2#3#4{{\fcolorbox{#1}{#2}{$\textcolor{#3}{ #4}$}}}
% border|background|text
\def\eqboxa#1{{\boxed{#1}}}
\def\eqboxb#1{{\eqbox{green}{white}{green}{#1}}}
\def\eqboxc#1{{\eqbox{blue}{white}{blue}{#1}}}
\def\eqboxd#1{{\eqbox{blue}{lightblue}{blue}{#1}}}

\def\ub{\mathbf{u}}
\def\wbopt{\mathop{\mathbf{w}}\limits^\triangle}
\def\deriv#1#2{\frac{\mathrm{d}#1}{\mathrm{d}#2}}
\def\Ub{\mathbf{U}}
\def\db{\mathbf{d}}
\def\eb{\mathbf{e}}
\def\yb{\mathbf{y}}
\def\ybp{\mathbf{y}}
\def\est#1{\hat{#1}}

\def\vb{\mathbf{v}}
\def\Ib{\mathbf{I}}
\def\Vb{\mathbf{V}}
\def\Lambdab{\mathbf{\Lambda}}
\def\Ab{\mathbf{A}}
\def\Bb{\mathbf{B}}
\def\Cb{\mathbf{C}}
\def\Db{\mathbf{D}}
\def\Kb{\mathbf{K}}

\def\textem#1{\emph{#1}}
%\def\url#1{\href{#1}{}}

\newtheorem{theorem}{Theorem}
\newtheorem{exercise}{Exercise}
\newtheorem{example}{Example}
\newtheorem{prop}{Property}
\newtheorem{remark}{Remark}
\newtheorem{proposition}{Proposition}
\newtheorem{definition}{Definition}


\def\url#1{\texttt{#1}}

\newenvironment{textboxa}
{ \begin{mdframed}[backgroundcolor=yellow]}
{ \end{mdframed} }





\tableofcontents
\newpage

\input{DelaysAndScales.tex}
\input{Intro_Filtering.tex}
\input{Intro_Fourier.tex}
\input{Fourier_transform.tex} %
\input{Discrete_Time_Fourier_properties.tex} %
\input{Convolution.tex}
\input{TransferFunctions.tex}
\input{Exercises_BasicSystemsRepr.tex} %
\input{BasicSystemsRepr.tex} %
\input{Continuous_time_case.tex} %
\input{Periodization_discretization.tex} 
\input{DFT.tex} 
\input{Sampling.tex} %

\input{LabImages_text.tex}
%\input{LabImages_correction.tex}

\input{DigitalFilters.tex}
\input{ZerosPoles.tex}
\input{FIR_synthesis.tex}
\input{Bilinear_synthesis.tex}


\input{BasicFiltering_text.tex}
%\input{BasicFiltering_correction.tex}



\chapter{Random Signals}
\input{Lecture1_RandomSignals}
\input{Lecture2_RandomSignals}
\input{Lecture3_RandomSignals}

%%Adaptive Filtering
\input{Optimum_filtering}
\input{Grad_algo}
\input{Adaptive_versions}
%\input{noisecancellationlab.tex}



\end{document}
